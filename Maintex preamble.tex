%Preamble for the main XeLaTeX file, for several language support
%formatting section
\documentclass[a4paper, twoside, openright, 10pt]{book}
\renewcommand{\sectionmark}[1]{}%Book specific, changes the header to blank, avoids really long titles as headers
\usepackage[subpreambles=true]{standalone}%standalone package for making the subfiles compile-able
\usepackage{mdframed}%nice boxes around the comment questions
\usepackage[a4paper, total={6in, 8in}]{geometry}%increases text size, haven't really thought much about this.
\providecommand{\tightlist}{\setlength{\itemsep}{0pt}\setlength{\parskip}{0pt}}%pandoc required.
%To include Parts on the TOC
\usepackage{tocloft}
\makeatletter
\@addtoreset{section}{part}
\makeatother
\newlength\mylen
\renewcommand\thepart{\arabic{part}}
\renewcommand\cftpartpresnum{Part~}
\settowidth\mylen{\bfseries\cftpartpresnum\cftpartaftersnum}
\addtolength\cftpartnumwidth{\mylen}

%XeLaTeX, fonts section
\usepackage{fontspec}%for XeLaTeX, selecting multiple fonts
\usepackage{polyglossia}%for XeLaTeX, enables multiple languages.
\setmainlanguage{english}
\setotherlanguages{arabic,hindi,sanskrit,greek,thai,russian,hebrew}

\setmainfont[Ligatures=TeX]{Latin Modern Roman}
\setsansfont[Ligatures=TeX]{Latin Modern Sans}
\setmonofont{Latin Modern Mono}

\newfontfamily\arabicfont[Script=Arabic]{Noto Naskh Arabic}
\newfontfamily\devanagarifont[Script=Devanagari]{Noto Serif Devanagari}
\newfontfamily\greekfont[Script=Greek]{GFS Artemisia}
\newfontfamily\thaifont[Script=Thai]{Noto Serif Thai}
\newfontfamily\cyrillicfont[Script=Cyrillic]{Noto Serif}
\newfontfamily\hebrewfont[Script=Hebrew]{Noto Serif Hebrew}

\usepackage[space]{xeCJK}
\setCJKmainfont{Noto Sans CJK SC}
\setCJKsansfont{Noto Sans CJK SC}
\setCJKmonofont{Noto Sans CJK SC}
\newCJKfontfamily\japanesefont{IPAexMincho}
\newCJKfontfamily\koreanfont{Baekmuk Batang}

\usepackage{ulem}%To enable strikethrough text, but emphasis is reversed to normal (italic) later, kinda redundant
\usepackage{changepage}
\usepackage{longtable}%pandoc requirement
\usepackage{booktabs}
\usepackage{import}
\usepackage{hyperref}
\hypersetup{pdftitle={/r/AskHistorians},colorlinks=false}
\newcommand\fnurl[2]{\href{#2}{#1}\footnote{\url{#2}}}$enables placing URLs as footnotes. Thanks to a user on StackExchange

